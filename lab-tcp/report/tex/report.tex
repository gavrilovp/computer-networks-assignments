\documentclass[a4paper,12pt]{article}

\input{../../../doc/preamble.tex}

\title{Отчёт по лабораторной работе \\ <<Механизмы протокола TCP>>}
\author{Гаврилов Павел Александрович}

\begin{document}

\maketitle

\tableofcontents

\section{Установка и разрыв соединения}

Устанавливаем соединение telnet-клиента на машине \textbf{c1} с
сервером на машине \textbf{c6}.

\begin{lstlisting}
c1:~# telnet 10.50.0.2 9
Trying 10.50.0.2...
Connected to 10.50.0.2.
Escape character is '^]'.
privet
^]
telnet> q
Connection closed.
\end{lstlisting}

Дамп на машине \textbf{c2} (интерфейс \textbf{eth0}):

\begin{lstlisting}
c2:~# tcpdump -n -i eth0 tcp
tcpdump: verbose output suppressed, use -v or -vv for full protocol decode
listening on eth0, link-type EN10MB (Ethernet), capture size 96 bytes
18:24:42.616886 IP 10.10.0.2.56174 > 10.50.0.2.9: S 3817357827:3817357827(0) win 5840 <mss 1460,sackOK,timestamp 1193 0,nop,wscale 1>
18:24:42.649317 IP 10.50.0.2.9 > 10.10.0.2.56174: S 3813612233:3813612233(0) ack 3817357828 win 5792 <mss 536,sackOK,timestamp 654 1193,nop,wscale 1>
18:24:42.649401 IP 10.10.0.2.56174 > 10.50.0.2.9: . ack 1 win 2920 <nop,nop,timestamp 1197 654>

18:26:51.173571 IP 10.10.0.2.56174 > 10.50.0.2.9: P 1:9(8) ack 1 win 2920 <nop,nop,timestamp 14051 654>
18:26:51.173826 IP 10.50.0.2.9 > 10.10.0.2.56174: . ack 9 win 2896 <nop,nop,timestamp 13506 14051>

18:27:15.797893 IP 10.10.0.2.56174 > 10.50.0.2.9: F 9:9(0) ack 1 win 2920 <nop,nop,timestamp 16513 13506>
18:27:15.800620 IP 10.50.0.2.9 > 10.10.0.2.56174: F 1:1(0) ack 10 win 2896 <nop,nop,timestamp 15969 16513>
18:27:15.800856 IP 10.10.0.2.56174 > 10.50.0.2.9: . ack 2 win 2920 <nop,nop,timestamp 16513 15969>
\end{lstlisting}

Дамп на машине \textbf{c3} (интерфейс \textbf{eth1}):
\begin{lstlisting}
3:~# tcpdump -n -i eth1 tcp
tcpdump: verbose output suppressed, use -v or -vv for full protocol decode
listening on eth1, link-type EN10MB (Ethernet), capture size 96 bytes
18:24:42.642762 IP 10.10.0.2.56174 > 10.50.0.2.9: S 3817357827:3817357827(0) win 5840 <mss 536,sackOK,timestamp 1193 0,nop,wscale 1>
18:24:42.652995 IP 10.50.0.2.9 > 10.10.0.2.56174: S 3813612233:3813612233(0) ack 3817357828 win 5792 <mss 536,sackOK,timestamp 654 1193,nop,wscale 1>
18:24:42.653148 IP 10.10.0.2.56174 > 10.50.0.2.9: . ack 1 win 2920 <nop,nop,timestamp 1197 654>
\end{lstlisting}

\section{Окно получателя}

Запустим программу \textbf{win} на машине \textbf{c4}:
\begin{lstlisting}
./win 3002 2 10
\end{lstlisting}

Запустим клиента на машине \textbf{c1}:
\begin{lstlisting}
netcat 10.30.0.2 3002 < /dev/zero
\end{lstlisting}

Запустим на машине \textbf{c3} программу \textbf{tcpdump}:
\begin{lstlisting}
окно получателя до нуля
c3:~# tcpdump -n -i eth0 tcp
tcpdump: verbose output suppressed, use -v or -vv for full protocol decode
listening on eth0, link-type EN10MB (Ethernet), capture size 96 bytes
19:10:03.753479 IP 10.10.0.2.52765 > 10.30.0.2.3002: S 1466136637:1466136637(0) win 5840 <mss 1460,sackOK,timestamp 2233873 0,nop,wscale 1>
19:10:03.764426 IP 10.30.0.2.3002 > 10.10.0.2.52765: S 1458145030:1458145030(0) ack 1466136638 win 2096 <mss 536,sackOK,timestamp 2233460 2233873,nop,wscale 1>
19:10:03.764574 IP 10.10.0.2.52765 > 10.30.0.2.3002: . ack 1 win 2920 <nop,nop,timestamp 2233876 2233460>
19:10:03.764668 IP 10.10.0.2.52765 > 10.30.0.2.3002: . 1:525(524) ack 1 win 2920 <nop,nop,timestamp 2233876 2233460>
19:10:03.764684 IP 10.10.0.2.52765 > 10.30.0.2.3002: . 525:1049(524) ack 1 win 2920 <nop,nop,timestamp 2233876 2233460>
19:10:03.764765 IP 10.30.0.2.3002 > 10.10.0.2.52765: . ack 525 win 1572 <nop,nop,timestamp 2233460 2233876>
19:10:03.764770 IP 10.30.0.2.3002 > 10.10.0.2.52765: . ack 1049 win 2096 <nop,nop,timestamp 2233460 2233876>
19:10:03.764885 IP 10.10.0.2.52765 > 10.30.0.2.3002: P 1049:1573(524) ack 1 win 2920 <nop,nop,timestamp 2233876 2233460>
19:10:03.764890 IP 10.10.0.2.52765 > 10.30.0.2.3002: . 1573:2097(524) ack 1 win 2920 <nop,nop,timestamp 2233876 2233460>
19:10:03.764893 IP 10.10.0.2.52765 > 10.30.0.2.3002: . 2097:2621(524) ack 1 win 2920 <nop,nop,timestamp 2233876 2233460>
19:10:03.764895 IP 10.10.0.2.52765 > 10.30.0.2.3002: P 2621:3145(524) ack 1 win 2920 <nop,nop,timestamp 2233876 2233460>
19:10:03.764971 IP 10.30.0.2.3002 > 10.10.0.2.52765: . ack 1573 win 2620 <nop,nop,timestamp 2233460 2233876>
19:10:03.764975 IP 10.30.0.2.3002 > 10.10.0.2.52765: . ack 2097 win 3144 <nop,nop,timestamp 2233460 2233876>
19:10:03.764978 IP 10.30.0.2.3002 > 10.10.0.2.52765: . ack 2621 win 3668 <nop,nop,timestamp 2233460 2233876>
19:10:03.764981 IP 10.30.0.2.3002 > 10.10.0.2.52765: . ack 3145 win 4192 <nop,nop,timestamp 2233460 2233876>
19:10:03.765090 IP 10.10.0.2.52765 > 10.30.0.2.3002: . 3145:3669(524) ack 1 win 2920 <nop,nop,timestamp 2233876 2233460>

19:10:04.130939 IP 10.10.0.2.52765 > 10.30.0.2.3002: . 78305:78829(524) ack 1 win 2920 <nop,nop,timestamp 2233914 2233497>
19:10:04.168548 IP 10.30.0.2.3002 > 10.10.0.2.52765: . ack 78829 win 36 <nop,nop,timestamp 2233501 2233914>
19:10:04.380755 IP 10.10.0.2.52765 > 10.30.0.2.3002: P 78829:78901(72) ack 1 win 2920 <nop,nop,timestamp 2233939 2233501>
19:10:04.380917 IP 10.30.0.2.3002 > 10.10.0.2.52765: . ack 78901 win 0 <nop,nop,timestamp 2233522 2233939>
19:10:04.610751 IP 10.10.0.2.52765 > 10.30.0.2.3002: . ack 1 win 2920 <nop,nop,timestamp 2233962 2233522>
19:10:04.610888 IP 10.30.0.2.3002 > 10.10.0.2.52765: . ack 78901 win 0 <nop,nop,timestamp 2233545 2233939>
19:10:05.060742 IP 10.10.0.2.52765 > 10.30.0.2.3002: . ack 1 win 2920 <nop,nop,timestamp 2234007 2233545>
19:10:05.060825 IP 10.30.0.2.3002 > 10.10.0.2.52765: . ack 78901 win 0 <nop,nop,timestamp 2233590 2233939>
19:10:05.768887 IP 10.30.0.2.3002 > 10.10.0.2.52765: . ack 78901 win 524 <nop,nop,timestamp 2233661 2233939>
19:10:05.769070 IP 10.10.0.2.52765 > 10.30.0.2.3002: P 78901:79353(452) ack 1 win 2920 <nop,nop,timestamp 2234076 2233661>
19:10:05.769081 IP 10.10.0.2.52765 > 10.30.0.2.3002: . 79353:79877(524) ack 1 win 2920 <nop,nop,timestamp 2234076 2233661>
19:10:05.769160 IP 10.30.0.2.3002 > 10.10.0.2.52765: R 1:1(0) ack 78901 win 524 <nop,nop,timestamp 2233661 2233939>
19:10:05.769284 IP 10.30.0.2.3002 > 10.10.0.2.52765: R 1458145031:1458145031(0) win 0
19:10:05.769290 IP 10.30.0.2.3002 > 10.10.0.2.52765: R 1458145031:1458145031(0) win 0
\end{lstlisting}

\section{Окно отправителя}

Устанавливаем на машине \textbf{c5} задержку канала 100мс.

\begin{lstlisting}
c5:~# /etc/delay 100
\end{lstlisting}

На машине \textbf{c2} запускаем перехват пакетов:
\begin{lstlisting}
c2:~# tcpdump -n -i eth0 tcp
tcpdump: verbose output suppressed, use -v or -vv for full protocol decode
listening on eth0, link-type EN10MB (Ethernet), capture size 96 bytes

c1:~# ./nagle nodelay 10.40.0.2 9 1 20
23:09:26.390111 IP 10.10.0.2.51445 > 10.40.0.2.9: S 3343984758:3343984758(0) win 5840 <mss 1460,sackOK,timestamp 3670139 0,nop,wscale 1>
23:09:26.498912 IP 10.40.0.2.9 > 10.10.0.2.51445: S 3340282288:3340282288(0) ack 3343984759 win 5792 <mss 1460,sackOK,timestamp 3669724 3670139,nop,wscale 1>
23:09:26.499022 IP 10.10.0.2.51445 > 10.40.0.2.9: . ack 1 win 2920 <nop,nop,timestamp 3670150 3669724>
23:09:26.499060 IP 10.10.0.2.51445 > 10.40.0.2.9: P 1:101(100) ack 1 win 2920 <nop,nop,timestamp 3670150 3669724>
23:09:26.505711 IP 10.10.0.2.51445 > 10.40.0.2.9: P 101:201(100) ack 1 win 2920 <nop,nop,timestamp 3670151 3669724>
23:09:26.608944 IP 10.40.0.2.9 > 10.10.0.2.51445: . ack 101 win 2896 <nop,nop,timestamp 3669735 3670150>
23:09:26.608953 IP 10.40.0.2.9 > 10.10.0.2.51445: . ack 201 win 2896 <nop,nop,timestamp 3669735 3670151>
23:09:26.609051 IP 10.10.0.2.51445 > 10.40.0.2.9: P 201:1201(1000) ack 1 win 2920 <nop,nop,timestamp 3670161 3669735>
23:09:26.615737 IP 10.10.0.2.51445 > 10.40.0.2.9: P 1201:1301(100) ack 1 win 2920 <nop,nop,timestamp 3670162 3669735>
23:09:26.619701 IP 10.10.0.2.51445 > 10.40.0.2.9: P 1301:1401(100) ack 1 win 2920 <nop,nop,timestamp 3670163 3669735>
23:09:26.629783 IP 10.10.0.2.51445 > 10.40.0.2.9: P 1401:1501(100) ack 1 win 2920 <nop,nop,timestamp 3670164 3669735>
23:09:26.689794 IP 10.10.0.2.51445 > 10.40.0.2.9: FP 1501:2001(500) ack 1 win 2920 <nop,nop,timestamp 3670169 3669735>
23:09:26.718949 IP 10.40.0.2.9 > 10.10.0.2.51445: . ack 1201 win 3896 <nop,nop,timestamp 3669746 3670161>
23:09:26.718957 IP 10.40.0.2.9 > 10.10.0.2.51445: . ack 1301 win 3896 <nop,nop,timestamp 3669746 3670162>
23:09:26.728937 IP 10.40.0.2.9 > 10.10.0.2.51445: . ack 1401 win 3896 <nop,nop,timestamp 3669746 3670163>
23:09:26.738921 IP 10.40.0.2.9 > 10.10.0.2.51445: . ack 1501 win 3896 <nop,nop,timestamp 3669748 3670164>
23:09:26.798904 IP 10.40.0.2.9 > 10.10.0.2.51445: F 1:1(0) ack 2002 win 4896 <nop,nop,timestamp 3669754 3670169>
23:09:26.799038 IP 10.10.0.2.51445 > 10.40.0.2.9: . ack 2 win 2920 <nop,nop,timestamp 3670179 3669754>
\end{lstlisting}

\section{Нейгл и Мишналь}

Мы должны увидеть, что мы посылаем неполный сегмент, как только
подтверждены все неполные сегменты (даже если полные не подтверждены).

\begin{lstlisting}
c1:~# ./nagle default 10.40.0.2 9 0 100

23:56:29.589836 IP 10.10.0.2.33412 > 10.40.0.2.9: S 371788788:371788788(0) win 5840 <mss 1460,sackOK,timestamp 3952458 0,nop,wscale 1>
23:56:29.818898 IP 10.40.0.2.9 > 10.10.0.2.33412: S 370276269:370276269(0) ack 371788789 win 5792 <mss 1460,sackOK,timestamp 3952044 3952458,nop,wscale 1>
23:56:29.819021 IP 10.10.0.2.33412 > 10.40.0.2.9: . ack 1 win 2920 <nop,nop,timestamp 3952482 3952044>
23:56:29.819077 IP 10.10.0.2.33412 > 10.40.0.2.9: P 1:101(100) ack 1 win 2920 <nop,nop,timestamp 3952482 3952044>
23:56:29.919907 IP 10.10.0.2.33412 > 10.40.0.2.9: . 101:1549(1448) ack 1 win 2920 <nop,nop,timestamp 3952493 3952044>
23:56:29.930023 IP 10.40.0.2.9 > 10.10.0.2.33412: . ack 101 win 2896 <nop,nop,timestamp 3952067 3952482>
23:56:29.930133 IP 10.10.0.2.33412 > 10.40.0.2.9: P 1549:1801(252) ack 1 win 2920 <nop,nop,timestamp 3952494 3952067>
\end{lstlisting}

\section{Аггрессивная буферизация}

Где что дампим.  Дампить без -t обязательно!

\begin{lstlisting}
c1:~# ./nagle cork 10.40.0.2 9 200 30

00:02:54.029696 IP 10.10.0.2.50841 > 10.40.0.2.9: P 1:301(300) ack 1 win 2920 <nop,nop,timestamp 3990904 3990419>
00:02:54.138952 IP 10.40.0.2.9 > 10.10.0.2.50841: . ack 301 win 3432 <nop,nop,timestamp 3990488 3990904>
00:02:54.709684 IP 10.10.0.2.50841 > 10.40.0.2.9: P 301:701(400) ack 1 win 2920 <nop,nop,timestamp 3990972 3990488>
00:02:54.818896 IP 10.40.0.2.9 > 10.10.0.2.50841: . ack 701 win 3968 <nop,nop,timestamp 3990556 3990972>
00:02:55.449767 IP 10.10.0.2.50841 > 10.40.0.2.9: P 701:1001(300) ack 1 win 2920 <nop,nop,timestamp 3991046 3990556>
00:02:55.559059 IP 10.40.0.2.9 > 10.10.0.2.50841: . ack 1001 win 4504 <nop,nop,timestamp 3990630 3991046>
00:02:56.039699 IP 10.10.0.2.50841 > 10.40.0.2.9: P 1001:1301(300) ack 1 win 2920 <nop,nop,timestamp 3991105 3990630>
00:02:56.148883 IP 10.40.0.2.9 > 10.10.0.2.50841: . ack 1301 win 4504 <nop,nop,timestamp 3990689 3991105>
00:02:56.629705 IP 10.10.0.2.50841 > 10.40.0.2.9: P 1301:1601(300) ack 1 win 2920 <nop,nop,timestamp 3991164 3990689>
00:02:56.738962 IP 10.40.0.2.9 > 10.10.0.2.50841: . ack 1601 win 4504 <nop,nop,timestamp 3990748 3991164>
00:02:57.209745 IP 10.10.0.2.50841 > 10.40.0.2.9: P 1601:1801(200) ack 1 win 2920 <nop,nop,timestamp 3991222 3990748>
00:02:57.318988 IP 10.40.0.2.9 > 10.10.0.2.50841: . ack 1801 win 4504 <nop,nop,timestamp 3990806 3991222>
00:02:57.689713 IP 10.10.0.2.50841 > 10.40.0.2.9: P 1801:2101(300) ack 1 win 2920 <nop,nop,timestamp 3991270 3990806>
00:02:57.798989 IP 10.40.0.2.9 > 10.10.0.2.50841: . ack 2101 win 4504 <nop,nop,timestamp 3990854 3991270>
00:02:58.219698 IP 10.10.0.2.50841 > 10.40.0.2.9: P 2101:2301(200) ack 1 win 2920 <nop,nop,timestamp 3991323 3990854>
00:02:58.328941 IP 10.40.0.2.9 > 10.10.0.2.50841: . ack 2301 win 4504 <nop,nop,timestamp 3990907 3991323>
00:02:58.689682 IP 10.10.0.2.50841 > 10.40.0.2.9: P 2301:2501(200) ack 1 win 2920 <nop,nop,timestamp 3991370 3990907>
00:02:58.798992 IP 10.40.0.2.9 > 10.10.0.2.50841: . ack 2501 win 4504 <nop,nop,timestamp 3990954 3991370>
00:02:59.139666 IP 10.10.0.2.50841 > 10.40.0.2.9: P 2501:2801(300) ack 1 win 2920 <nop,nop,timestamp 3991415 3990954>
00:02:59.249078 IP 10.40.0.2.9 > 10.10.0.2.50841: . ack 2801 win 4504 <nop,nop,timestamp 3990999 3991415>
00:02:59.659736 IP 10.10.0.2.50841 > 10.40.0.2.9: P 2801:3001(200) ack 1 win 2920 <nop,nop,timestamp 3991467 3990999>
00:02:59.739901 IP 10.10.0.2.50841 > 10.40.0.2.9: F 3001:3001(0) ack 1 win 2920 <nop,nop,timestamp 3991475 3990999>
00:02:59.768905 IP 10.40.0.2.9 > 10.10.0.2.50841: . ack 3001 win 4504 <nop,nop,timestamp 3991051 3991467>
00:02:59.848859 IP 10.40.0.2.9 > 10.10.0.2.50841: F 1:1(0) ack 3002 win 4504 <nop,nop,timestamp 3991059 3991475>
00:02:59.848927 IP 10.10.0.2.50841 > 10.40.0.2.9: . ack 2 win 2920 <nop,nop,timestamp 3991484 3991059>
\end{lstlisting}

\section{Отправка без задержки}

\begin{lstlisting}
c1:~# ./nagle nodelay 10.40.0.2 9 0 30

c2:~# tcpdump -n -i eth0 tcp

00:02:29.637284 IP 10.10.0.2.54489 > 10.40.0.2.9: P 1:101(100) ack 1 win 2920 <nop,nop,timestamp 954429 2027455>
00:02:29.637304 IP 10.10.0.2.54489 > 10.40.0.2.9: P 101:201(100) ack 1 win 2920 <nop,nop,timestamp 954429 2027455>
00:02:29.747144 IP 10.40.0.2.9 > 10.10.0.2.54489: . ack 101 win 2896 <nop,nop,timestamp 2027466 954429>
00:02:29.747163 IP 10.40.0.2.9 > 10.10.0.2.54489: . ack 201 win 2896 <nop,nop,timestamp 2027466 954429>
00:02:29.747332 IP 10.10.0.2.54489 > 10.40.0.2.9: . 201:1649(1448) ack 1 win 2920 <nop,nop,timestamp 954440 2027466>
\end{lstlisting}

\section{Быстрый повтор}

Запустим на машине \textbf{c3} сценарий, отбрасывающий каждый 10-ый пакет.

\begin{lstlisting}
c3:~# /etc/drop 10
I will drop every 10 packet (counting from 1).
\end{lstlisting}

Запустим tcpdump на \textbf{c2}.

\begin{lstlisting}
20:52:35.922486 IP 10.10.0.2.49641 > 10.30.0.2.9: P 1:101(100) ack 1 win 2920 <nop,nop,timestamp 888526 888194>
20:52:35.922571 IP 10.30.0.2.9 > 10.10.0.2.49641: . ack 101 win 1048 <nop,nop,timestamp 888194 888526>
20:52:35.932575 IP 10.10.0.2.49641 > 10.30.0.2.9: P 101:201(100) ack 1 win 2920 <nop,nop,timestamp 888527 888194>
20:52:35.932747 IP 10.30.0.2.9 > 10.10.0.2.49641: . ack 201 win 1048 <nop,nop,timestamp 888194 888527>
20:52:35.942135 IP 10.10.0.2.49641 > 10.30.0.2.9: P 201:301(100) ack 1 win 2920 <nop,nop,timestamp 888528 888194>
20:52:35.942277 IP 10.30.0.2.9 > 10.10.0.2.49641: . ack 301 win 1048 <nop,nop,timestamp 888197 888528>
20:52:35.952100 IP 10.10.0.2.49641 > 10.30.0.2.9: P 301:401(100) ack 1 win 2920 <nop,nop,timestamp 888529 888197>
20:52:35.961895 IP 10.10.0.2.49641 > 10.30.0.2.9: P 401:501(100) ack 1 win 2920 <nop,nop,timestamp 888529 888197>
20:52:35.962064 IP 10.30.0.2.9 > 10.10.0.2.49641: . ack 301 win 1048 <nop,nop,timestamp 888198 888528,nop,nop,sack 1 {401:501}>
20:52:35.972197 IP 10.10.0.2.49641 > 10.30.0.2.9: P 501:601(100) ack 1 win 2920 <nop,nop,timestamp 888531 888198>
20:52:35.972499 IP 10.30.0.2.9 > 10.10.0.2.49641: . ack 301 win 1048 <nop,nop,timestamp 888200 888528,nop,nop,sack 1 {401:601}>
20:52:35.982461 IP 10.10.0.2.49641 > 10.30.0.2.9: P 601:701(100) ack 1 win 2920 <nop,nop,timestamp 888532 888200>
20:52:35.982686 IP 10.30.0.2.9 > 10.10.0.2.49641: . ack 301 win 1048 <nop,nop,timestamp 888201 888528,nop,nop,sack 1 {401:701}>
20:52:35.982791 IP 10.10.0.2.49641 > 10.30.0.2.9: P 301:401(100) ack 1 win 2920 <nop,nop,timestamp 888532 888201>
20:52:35.982947 IP 10.30.0.2.9 > 10.10.0.2.49641: . ack 701 win 1048 <nop,nop,timestamp 888201 888532>
20:52:35.992564 IP 10.10.0.2.49641 > 10.30.0.2.9: P 701:801(100) ack 1 win 2920 <nop,nop,timestamp 888533 888201>
\end{lstlisting}

\section{Обычный повтор}

Также, как и выше, но посылаем меньше пакетов, чтобы не получить 4 одинаковых подтверждения.

\begin{lstlisting}
21:03:27.142119 IP 10.30.0.2.9 > 10.10.0.2.56121: . ack 301 win 1048 <nop,nop,timestamp 953317 953648>
21:03:27.152023 IP 10.10.0.2.56121 > 10.30.0.2.9: P 301:401(100) ack 1 win 2920 <nop,nop,timestamp 953649 953317>
21:03:27.161953 IP 10.10.0.2.56121 > 10.30.0.2.9: P 401:501(100) ack 1 win 2920 <nop,nop,timestamp 953649 953317>
21:03:27.162067 IP 10.30.0.2.9 > 10.10.0.2.56121: . ack 301 win 1048 <nop,nop,timestamp 953318 953648,nop,nop,sack 1 {401:501}>
21:03:27.172176 IP 10.10.0.2.56121 > 10.30.0.2.9: F 501:501(0) ack 1 win 2920 <nop,nop,timestamp 953651 953318>
21:03:27.172295 IP 10.30.0.2.9 > 10.10.0.2.56121: . ack 301 win 1048 <nop,nop,timestamp 953320 953648,nop,nop,sack 1 {401:502}>
21:03:27.372022 IP 10.10.0.2.56121 > 10.30.0.2.9: P 301:401(100) ack 1 win 2920 <nop,nop,timestamp 953671 953320>
21:03:27.372213 IP 10.30.0.2.9 > 10.10.0.2.56121: . ack 502 win 1048 <nop,nop,timestamp 953339 953671>
21:03:27.372853 IP 10.30.0.2.9 > 10.10.0.2.56121: F 1:1(0) ack 502 win 1048 <nop,nop,timestamp 953339 953671>
21:03:27.372934 IP 10.10.0.2.56121 > 10.30.0.2.9: . ack 2 win 2920 <nop,nop,timestamp 953671 953339>
\end{lstlisting}

\section{Опыт с PMTU}

Для опыта нужно на c3 и c4 отключить tcpclump.
Надеюсь тут поможет команда iptables -F.
Для сброса кеша MSS на с1 его можно тупо перегрузить.

После этих подготовительных действий мы наверное увидим pmtu в действии при попытке соединится с c1 на c6.

Дампим (tcp) на c2 eth0.

\begin{lstlisting}
22:58:10.254345 IP 10.10.0.2.57817 > 10.50.0.2.9: S 2798706221:2798706221(0) win 5840 <mss 1460,sackOK,timestamp 568490 0,nop,wscale 1>
22:58:10.254694 IP 10.50.0.2.9 > 10.10.0.2.57817: S 3395826797:3395826797(0) ack 2798706222 win 5792 <mss 1460,sackOK,timestamp 1641414 568490,nop,wscale 1>
22:58:10.254769 IP 10.10.0.2.57817 > 10.50.0.2.9: . ack 1 win 2920 <nop,nop,timestamp 568490 1641414>
22:58:10.393071 IP 10.10.0.2.57817 > 10.50.0.2.9: . 1:1449(1448) ack 1 win 2920 <nop,nop,timestamp 568505 1641414>
22:58:10.393091 IP 10.10.0.1 > 10.10.0.2: ICMP 10.50.0.2 unreachable - need to frag (mtu 1492), length 556
22:58:10.393237 IP 10.10.0.2.57817 > 10.50.0.2.9: . 1:1441(1440) ack 1 win 2920 <nop,nop,timestamp 568505 1641414>
22:58:10.393246 IP 10.10.0.2.57817 > 10.50.0.2.9: . 1441:1449(8) ack 1 win 2920 <nop,nop,timestamp 568505 1641414>
22:58:10.393340 IP 10.20.0.2 > 10.10.0.2: ICMP 10.50.0.2 unreachable - need to frag (mtu 576), length 556
22:58:10.393472 IP 10.50.0.2.9 > 10.10.0.2.57817: . ack 1 win 2896 <nop,nop,timestamp 1641428 568490,nop,nop,sack 1 {1441:1449}>
22:58:10.393536 IP 10.10.0.2.57817 > 10.50.0.2.9: . 1:525(524) ack 1 win 2920 <nop,nop,timestamp 568505 1641428>
22:58:10.393542 IP 10.10.0.2.57817 > 10.50.0.2.9: . 525:1049(524) ack 1 win 2920 <nop,nop,timestamp 568505 1641428>
22:58:10.393714 IP 10.50.0.2.9 > 10.10.0.2.57817: . ack 525 win 3432 <nop,nop,timestamp 1641428 568505,nop,nop,sack 1 {1441:1449}>
22:58:10.393718 IP 10.50.0.2.9 > 10.10.0.2.57817: . ack 1049 win 3956 <nop,nop,timestamp 1641428 568505,nop,nop,sack 1 {1441:1449}>
22:58:10.393767 IP 10.10.0.2.57817 > 10.50.0.2.9: . 1049:1441(392) ack 1 win 2920 <nop,nop,timestamp 568505 1641428>
22:58:10.393896 IP 10.50.0.2.9 > 10.10.0.2.57817: . ack 1449 win 4480 <nop,nop,timestamp 1641428 568505>
22:58:10.403856 IP 10.10.0.2.57817 > 10.50.0.2.9: FP 1449:1501(52) ack 1 win 2920 <nop,nop,timestamp 568506 1641428>
22:58:10.404693 IP 10.50.0.2.9 > 10.10.0.2.57817: F 1:1(0) ack 1502 win 4480 <nop,nop,timestamp 1641430 568506>
22:58:10.404895 IP 10.10.0.2.57817 > 10.50.0.2.9: . ack 2 win 2920 <nop,nop,timestamp 568506 1641430>
\end{lstlisting}

\section{Соединение с неверным портом}

\begin{lstlisting}
21:09:42.392923 IP 10.10.0.2.57332 > 10.30.0.2.8: S 144902252:144902252(0) win 5840 <mss 1460,sackOK,timestamp 991172 0,nop,wscale 1>
21:09:42.393062 IP 10.30.0.2.8 > 10.10.0.2.57332: R 0:0(0) ack 144902253 win 0
\end{lstlisting}

\end{document}
